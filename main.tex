\documentclass[aspectratio=169]{beamer}
\usetheme[background=light]{moloch}

\usepackage[backend=biber,style=authortitle,dashed=false,url=false]{biblatex}
\usepackage{svg}
\usepackage{tabularx}
\usepackage{emoji}
\usepackage{minted}
\usepackage{ragged2e}
\usepackage{xmpmulti}
\usepackage{caption}

\definecolor{ref}{HTML}{000080}
\definecolor{links}{HTML}{0066CC}
\hypersetup{colorlinks,linkcolor=,citecolor=ref,urlcolor=links}

\addbibresource{references.bib}

\title{C++20 Modules Support in SonarQube}
\subtitle{How We Accidentally Became a Build System}
\date{`using std::cpp', March 19, 2025}
\author{Alejandro Álvarez Ayllón}
\titlegraphic{\vbox to 3em {\hfill\includesvg[height=2em]{logos/SonarDark.svg}}}

% No section nor subsection slides
%\AtBeginSection[]{}
%\AtBeginSubsection[]{}

\setminted{fontsize=\scriptsize}

\newbibmacro{string+url}[1]{%
  \iffieldundef{url}{#1}{\href{\thefield{url}}{#1}}}
\DeclareFieldFormat{title}{\usebibmacro{string+url}{\mkbibemph{#1}}}
\DeclareFieldFormat[unpublished]{title}{\usebibmacro{string+url}{\mkbibquote{#1}}}
\DeclareFieldFormat[misc]{title}{\usebibmacro{string+url}{\mkbibquote{#1}}}

\begin{document}

\begin{frame}[standout]
  \setbeamercolor{title}{bg=black, fg=white}
  \setbeamercolor{subtitle}{fg=lightgray}
  \setbeamercolor{author}{fg=white}
  \setbeamercolor{date}{fg=white}
  \titlepage
\end{frame}

\begin{frame}{About Me}
  \begin{block}{Alejandro Álvarez Ayllón \emoji{flag-spain}}
    Completed my Master's and PhD at the University of Cádiz.
  \end{block}
  \begin{block}{Now Staff Engineer at Sonar}
    \begin{itemize}
      \item Member of ``CFamily'' \textit{squad}.
      \item Contributed to the support of C++20 modules.
    \end{itemize}
  \end{block}

  \begin{block}{Previously...}
    \begin{itemize}
      \item University of Geneva: Astronomy Department (Euclid consortium).
      \item CERN: File Transfer Service (FTS) for LHC experiments.
    \end{itemize}
  \end{block}
\end{frame}

\begin{frame}{About Sonar}
  \begin{block}{Sonar}
    \begin{itemize}
      \item Continuous inspection of code quality and security
      \item Supports >30 languages
      \item \approx 700 employees \emoji{flag-switzerland} \emoji{flag-united-states} \emoji{flag-united-kingdom} \emoji{flag-singapore} \emoji{flag-germany} \emoji{flag-france}
      \item SonarQube \\
            \vspace{0.5em}
            \small \begin{tabular}{l l r}
              \href{https://www.sonarsource.com/open-source-editions/sonarqube-community-edition/}{Community} & Free and Open-source Platform                 &                                                           \\ \vspace{0.2em}
              \href{https://www.sonarsource.com/products/sonarqube/}{Server}                                  & On-premises installation                      & \raisebox{-.25\height}{\includesvg[width=1em]{logos/C++}} \\ \vspace{0.2em}
              \href{https://sonarcloud.io/login}{Cloud}                                                       & Cloud-based service, free for public projects & \raisebox{-.25\height}{\includesvg[width=1em]{logos/C++}} \\ \vspace{0.2em}
              \href{https://www.sonarsource.com/products/sonarlint/}{For IDE}                                 & VSCode, Visual Studio, IntelliJ, Eclipse      & \raisebox{-.25\height}{\includesvg[width=1em]{logos/C++}} \\
            \end{tabular}
    \end{itemize}
  \end{block}
\end{frame}

\begin{frame}{What to Expect}
  \begin{block}{}
    \begin{itemize}
      \item A retrospective on \emph{our} experiences adding C++20 modules support.
      \item Not a tutorial on C++20 modules.
            \begin{itemize}
              \item \cite{Weis24}
              \item I will give an overview of them, but I will not go into details.
            \end{itemize}
    \end{itemize}
  \end{block}
\end{frame}

\section{Brief Refresher on C++ Modules}
\begin{frame}[t]{Before C++20}
  \setbeamercovered{transparent}
  \begin{columns}
    \begin{column}{0.5\textwidth}
      \onslide<1>{\begin{block}{library.h}
          \inputminted{cpp}{snippets/cpp17/library.h}
        \end{block}}
      \onslide<2>{\begin{block}{library.cpp}
          \inputminted{cpp}{snippets/cpp17/library.cpp}
        \end{block}}
    \end{column}
    \begin{column}{0.5\textwidth}
      \onslide<3>{%
        \begin{block}{main.cpp}
          \inputminted{cpp}{snippets/cpp17/main.cpp}
        \end{block}
      }
    \end{column}
  \end{columns}
  \begin{block}{}
    \onslide<4>{%
      \inputminted{bash}{snippets/cpp17/build.sh}
    }
  \end{block}

  \note[item]{Compilation units are independent.}
  \note[item]{Compilation is parallelizable.}
  \note[item]{If there were more TU depending on \texttt{library.h}, they could be compiled in parallel.}
  \note[item]{But library.h has to be parsed every time, including its string include.}
\end{frame}


\begin{frame}{Since C++20}
  \setbeamercovered{transparent}
  \begin{columns}
    \onslide<1>{\begin{column}{0.5\textwidth}
      \begin{block}{library.cppm}
        \inputminted{cpp}{snippets/cpp20/library.cppm}
      \end{block}}
    \end{column}
    \begin{column}{0.5\textwidth}
      \onslide<2>{\begin{block}{main.cpp}
          \inputminted{cpp}{snippets/cpp20/main.cpp}
        \end{block}}
    \end{column}
  \end{columns}
  \begin{block}{}
    \onslide<3>{\inputminted{bash}{snippets/cpp20/build.sh}}
  \end{block}

  \note[item]{Now, if library is used from multiple TUs, it is parsed only once.}
\end{frame}

% Some main differences

% Terminology
\begin{frame}[fragile]{Terminology}
  \setbeamercovered{transparent}
  \begin{columns}
    \begin{column}{0.5\textwidth}

      \begin{block}<+>{\footnotesize Global Module Fragment}
        \begin{minted}{cpp}
module;
// THIS is the GMF
// Only preprocessor directives
[export] module <identifier>;
    \end{minted}
      \end{block}

      \begin{block}<+>{\footnotesize Primary Module Interface Unit}
        \begin{minted}{cpp}
export module <identifier>;
// Declarations
    \end{minted}
      \end{block}

      \begin{block}<+>{\footnotesize Module Implementation Unit}
        \begin{minted}{cpp}
module <identifier>;

export import :part1;
export import :part2;
// Declarations
        \end{minted}
      \end{block}
    \end{column}

    \begin{column}{0.5\textwidth}

      \begin{block}<+>{\footnotesize Module Interface Partition}
        \begin{minted}{cpp}
export module <identifier>:part1;
// Declarations
        \end{minted}
      \end{block}

      \begin{block}<+>{\footnotesize Private Module Fragment}
        \begin{minted}{cpp}
export module <identifier>;
// Declarations

module : private;
// _Unreachable_ Declarations
        \end{minted}
      \end{block}

    \end{column}

  \end{columns}
\end{frame}

\begin{frame}[t,fragile]{Terminology}

  \begin{columns}[t]
    \begin{column}{0.3\textwidth}
      \begin{block}{\footnotesize Attachment}
        \justifying \footnotesize If a declaration appears after the module declaration, it is attached to the module.
      \end{block}
    \end{column}

    \begin{column}{0.3\textwidth}
      \begin{block}{\footnotesize Module Linkage}
        \justifying   \footnotesize If attached to a named module but not exported.
      \end{block}
    \end{column}

    \begin{column}{0.3\textwidth}
      \begin{block}{\footnotesize Ownership Model}
        \justifying \footnotesize \textbf{Strong} if the module is part of the mangled name. \textbf{Weak} otherwise.
      \end{block}
    \end{column}
  \end{columns}

  \vspace{0.5em}

  \begin{columns}
    \begin{column}{0.5\textwidth}
      \begin{minted}{cpp}
export module parser;

// Attached to parser.
// Module linkage, `flags@parser`
int flags();

// Attached to parser.
// External linkage.
// `::version@parser`, strong ownership
// `::version`, weak ownership
export int version();
\end{minted}
    \end{column}
    \begin{column}{0.5\textwidth}
      \begin{minted}{cpp}
export module serializer;

// Attached to serializer.
// Module linkage, `flags@serializer`
int flags();

// Attached to serializer.
// External linkage.
// `::version@serializer` => OK
// `::version` => ODR violation
export int version();
  \end{minted}
    \end{column}
  \end{columns}
\end{frame}

\begin{frame}[fragile]{Terminology}
  \begin{columns}
    \begin{column}{0.5\textwidth}
      \begin{block}{\footnotesize Visibility}
        \footnotesize Can it be named from outside?
      \end{block}
    \end{column}
    \begin{column}{0.5\textwidth}
      \begin{block}{\footnotesize Reachability}
        \footnotesize Can it be reached from a visibible declaration?
      \end{block}
    \end{column}

  \end{columns}

  \vspace{0.5em}

  \begin{minted}{cpp}
export module parser;

export const int version;      // Visible

int internal () { return 0; }; // Not visible, module linkage

struct Handler {};             // Not visible, but reachable

export namespace parser {      // Visible
Handler create_parser() { return Handler {}; } // Visible, external linkage
}
  \end{minted}

  \vspace{0.5em}

  \pause
  \small \cite{Engert21}

\end{frame}

\begin{frame}{Terminology}
  \begin{block}{}
    \textbf{BMI} Binary Module Interface \\
    \textbf{CMI} Compiled Module Interface

    \vspace{1em}

    Basically  a memory dump of the compiler internal representation of a module AST.

    They need to be built before the module can be imported.

    They are \emph{not} portable. Compiler, version, and even flags can affect them.
  \end{block}
\end{frame}

\begin{frame}{File Name and Extensions}
  File names and extensions for module units are not significant; they are a matter of convention.

  \begin{itemize}
    \item \texttt{module.cppm} is preferred by clang.
    \item \texttt{module.cpp} is preferred by gcc.
    \item \texttt{module.ixx} is preferred by msvc.
  \end{itemize}

  But they don't matter much if the right flags are used.
  \note[item]{The extension is used by the compilers to automatically enable the corresponding flags.}
  \note[item]{This is relevant for the analyzer as well! We can not rely on the extension.}
\end{frame}

\begin{frame}[fragile]{Terminology}
  \begin{block}{Header Units}
    \begin{minted}[fontsize=\normalsize]{cpp}
import <iostream>;
    \end{minted}
    \vspace{1em}
    Hard to use, hard to implement.
    \begin{itemize}
      \item \cite{Engert23}
      \item \cite{Ruoso23}
    \end{itemize}
  \end{block}
\end{frame}

% Requires build-system support (dependencies)
\begin{frame}{Build System Support}
  C++20 modules require build system support.
  \begin{itemize}
    \item Source files need to be scanned to find the dependencies.
    \item Modules need to be built before they can be imported.
    \item Compilation is \emph{not} embarrassingly parallel anymore.
  \end{itemize}
\end{frame}

% Current status of compilers, libraries, etc.
\begin{frame}{Current Status}
  \begin{columns}[t]
    \begin{column}{0.5\textwidth}
      \begin{block}{Compilers}
        \small
        \begin{description}
          \item[GCC] Partial \note[item]{i.e. GCC does not implement private fragment}
          \item[Clang] Partial
          \item[MSVC] Complete
          \item[Apple Clang] No
        \end{description}
        It is recommended to use the latest version of your compiler.
      \end{block}
    \end{column}

    \begin{column}{0.5\textwidth}
      \begin{block}{\texttt{import std;}}
        \small
        \begin{description}
          \item[libstdc++] Partial (\ge 15)
          \item[libc++] Partial (\ge 17)
          \item[MSSTL] Yes (\ge 17.5)
        \end{description}
      \end{block}
    \end{column}
  \end{columns}

  \vspace{1em}

  \begin{block}{Build Systems}
    \href{https://build2.org/}{build2}, \href{https://cmake.org/}{CMake},
    \href{https://learn.microsoft.com/en-us/visualstudio/msbuild/msbuild-command-line-reference?view=vs-2022}{MSBuild}
    and \href{https://xmake.io/}{xmake} support modules.
  \end{block}
\end{frame}

% We take code, analyzer it (duh!)
% Need to build the context.
% Tease about autoscan
\begin{frame}[t]{About the CFamily Analyzer}
  \begin{block}{}
    \begin{itemize}
      \item C, C++, Objective-C.
      \item Built on top of a fork of \href{https://clang.llvm.org/}{Clang} (19 as of today).
      \item We analyze the code \emph{how it was compiled}. We need to know
            \begin{itemize}
              \item The Compiler.
              \item The C++ standard.
              \item The include paths.
              \item The flags.
              \item The defines.
            \end{itemize}
      \item We leverage \texttt{compile\_commands.json}
            \begin{itemize}
              \item Generated by the build tools.
              \item Or by the \texttt{build-wrapper}.
            \end{itemize}
      \item We can also figure out some of these things \href{https://docs.sonarsource.com/sonarqube-cloud/advanced-setup/automatic-analysis}{automatically}. \note{This will be relevant later}
    \end{itemize}
  \end{block}
  \note[item]{PVS Studio has \texttt{pvs-studio-analyzer trace}}
  \note[item]{Coverity has \texttt{cov-build}}
  \note[item]{Klocwork has \texttt{kwinject}}
\end{frame}

% We need to know how it is compiled (why?)
\begin{frame}[fragile]{\texttt{compile\_commands.json}}
  \inputminted[breaklines]{json}{snippets/compdb/compile_commands.json}
  \begin{minted}{bash}
# These are generated when building
# std.cppm.o.modmap
-x c++-module -fmodule-output=CMakeFiles/__cmake_cxx23.dir/std.pcm
# main.cpp.o.modmap
-fmodule-file=std=CMakeFiles/__cmake_cxx23.dir/std.pcm
  \end{minted}
  \note{Wait, how do we generate the map? We need that. Hold that thought.}
\end{frame}

% Compilation database
% Build Wrapper

\section*{Let's Add Support for Modules!}

% Why now
%   Compiler & build system implementations starting to be usable, and converge on behavior (i.e. strong ownership)
%   We didn't want to implement something that could break in a month
%   We see low adoption, but of course, we do not support it. Nonetheless, not many requests either.
%   Anyway, let's add to the momentum, it seems a good moment.
\begin{frame}{Why now?}
  \begin{itemize}
    \item CMake support became non-experimental in version 3.28 (October 2023).
    \item Clang 19 was to be branched in Summer 2024.
          \begin{itemize}
            \item We need \texttt{-fmodules-reduced-bmi} for the feature to be usable.
            \item It purges unreachable entities from the BMIs.
          \end{itemize}
    \item The implementations are starting to be usable and stabilize.
  \end{itemize}
\end{frame}

\begin{frame}{Projects to test}
  \begin{itemize}
    \item \href{https://tinyurl.com/rwd48zpb}{SourceGraph query} for \texttt{export  module}, 87 hits.
    \item \href{https://tinyurl.com/ejey2pnt}{SourceGraph query} for \texttt{import}, 48 hits.
    \item Not great, many are toy examples, or test suites.
          \begin{itemize}
            \item i.e. gcc or llvm.
          \end{itemize}
  \end{itemize}
  \begin{figure}
    \includegraphics[width=\linewidth]{arewemodules.png}
    \caption{\href{https://arewemodulesyet.org/}{Are we modules yet?}}
  \end{figure}
\end{frame}

\begin{frame}{Projects to test}
  Nonetheless, we found some interesting ones:
  \begin{itemize}
    \item \href{https://github.com/infiniflow/infinity}{Infinity}, and AI database.
    \item \href{https://github.com/mpusz/mp-units}{mp-units}, quantities and units library (proposed for C++29).
    \item \href{https://github.com/alibaba/async_simple/tree/CXX20Modules}{async\_simple}, library of asynchronous components.
    \item And a few others.
  \end{itemize}
\end{frame}

% Dependencies
\begin{frame}{First things first}
  We need to know who exports and imports what.
  \begin{itemize}
    \item File names and extensions are not meaningful.
    \item Can we use the compilation database?
  \end{itemize}
\end{frame}

\begin{frame}{First things first}
  \begin{block}{msvc \emoji{smile}}
    \footnotesize{
    \texttt{/TP /interface /ifcOutput {\color{red} mod.pcm} /Fo:mod.obj {\color{blue}mod-source.cppm}} \\
    \texttt{/reference {\color{purple}mod=mod.pcm} /Fo:main.obj {\color{blue}main.cpp}}
    }
  \end{block}
  \begin{block}{clang \emoji{smile}}
    \footnotesize{
    \texttt{-x c++-module -fmodule-output={\color{red}mod.pcm} -o mod.o {\color{blue}mod-source.cppm}} \\
    \texttt{-fmodule-file={\color{purple}mod=mod.pcm} -o main.o {\color{blue}main.cpp}}
    }
  \end{block}
  \begin{block}{gcc \emoji{neutral-face}}
    \footnotesize{
    \texttt{-fmodule-mapper={\color{magenta}mod.mm} -o mod.o -x c++ {\color{blue}mod-source.cppm}} \\
    \texttt{-fmodule-mapper={\color{magenta}mod.mm} -o main.o {\color{blue} main.cpp}} \\
    {\color{teal}\texttt{\# Content of `mod.mm'}} \\
    \texttt{mod mod.gcm}
    }
  \end{block}
\end{frame}

\begin{frame}{First things first}
  \begin{block}{msvc \emoji{unamused-face}}
    \footnotesize{
    \texttt{/TP /interface {\color{magenta}/ifcOutput modules} /Fo:mod.obj {\color{blue} mod-source.cppm}} \\
    \texttt{{\color{magenta}/ifcSearchDir modules} /Fo:main.obj main.cpp}
    }
  \end{block}
  \begin{block}{clang \emoji{unamused-face}}
    \footnotesize{
    \texttt{-x c++-module -fmodule-output={\color{red}mod.pcm} -o mod.o {\color{blue}mod-source.cppm}} \\
    \texttt{{\color{magenta}-fprebuilt-module-path=.} main.cpp}
    }
  \end{block}
  \begin{block}{gcc \emoji{face-exhaling}}
    \footnotesize{
    \texttt{-c -x c++ -o mod.o mod-source.cppm} \\
    \texttt{-c main.cpp} \\
    {\color{teal}\texttt{\# BMI created implicitly under \texttt{gcm.cache/mod.gcm}}}
    }
  \end{block}
\end{frame}

%   Forget it, scan dependencies!
\begin{frame}{First things first}
  We'll scan the dependencies ourselves.
  \begin{itemize}
    \item Involves only the preprocessor, so it can be done quickly.
    \item We can do it in parallel.
  \end{itemize}
  \begin{table}
    \begin{tabular}{l r l}
                    & \textbf{TU's} &              \\
      Infinity      & 1\,485        & 7 seconds    \\
      async\_simple & 87            & 0.45 seconds
    \end{tabular}
    \caption*{With 16 threads}
  \end{table}
  \note{We have re-invented \texttt{clang-scan-deps}, which is what CMake uses for clang and modules.}
\end{frame}

\begin{frame}[fragile]{First things first}
  After scanning, we have a list of who-provides-what, and who-needs-what.
  \begin{block}{}
    \footnotesize{
    \texttt{{\color{blue}stl}=src/common/stl.cppm: []\\
    {\color{blue}default\_values}=src/common/default\_values.cppm: [{\color{purple}stl}]\\
    {\color{blue}ring\_buffer\_iterator}=src/network/ring\_buffer\_iterator.cppm: [{\color{purple}stl, default\_values}]\\
    ...}
    }
  \end{block}
  We need to put together a graph.
\end{frame}

\begin{frame}{Dependency Graph}
  \begin{figure}
    \includegraphics[width=0.8\textwidth]{graph/ptraced.pdf}
    \caption*{\footnotesize From a pet project, so it fits on a slide.}
  \end{figure}
\end{frame}

\begin{frame}{Building the BMIs}
  \multiinclude[<+->][format=pdf, graphics={width=0.9\textwidth}]{graph/building/step}
\end{frame}

\begin{frame}{We can now analyze}
  \begin{itemize}
    \item We could analyze as we build the BMIs, but we decided to do it in two steps.
    \item Once we have the BMIs, we can parallelize the analysis to the maximum, which is more intensive.
  \end{itemize}
  \begin{block}{What do we do with the BMIs?}
    Infinity's take 2.61 GiB of disk space. Uploading/downloading them from the server is not a good option.

    They are transient, and we rebuild them when necessary.
    \note{And that's with the reduced BMI!}
  \end{block}
\end{frame}

\begin{frame}{Do we need to build everything every time?}
  \begin{block}{Incremental analysis}
    The CFamily analyzer does not analyze the whole project every time.
    Neither on the \texttt{main} branch, nor on PRs.
    \begin{itemize}
      \item We analyze only the files that changed.
      \item If a header changes, we analyze the files that include it.
    \end{itemize}
    We need to adapt this to our handling of modules.
  \end{block}
\end{frame}

\begin{frame}{Incremental analysis}
  \multiinclude[<+->][format=pdf, end=2,graphics={width=0.9\textwidth}]{graph/changed/step}
\end{frame}

\begin{frame}{Incremental analysis \emoji{see-no-evil-monkey}}
  \includegraphics[width=0.9\textwidth]{graph/changed/step-3.pdf}
\end{frame}


\begin{frame}{Incremental analysis}
  \begin{itemize}
    \item We ended up rebuilding almost everything.
    \item We do \emph{not} reanalyze everything.
    \item Changes to deeply nested modules can trigger a cascade of rebuilds.
    \item On the bright side
          \begin{table}
            \begin{tabular}{l r l}
                            & \textbf{BMIs} &           \\
              Infinity      & 643           & 2 minutes \\
              async\_simple & 64            & 5 seconds
            \end{tabular}
          \end{table}
  \end{itemize}
\end{frame}

\begin{frame}{Overhead summary}
  \footnotesize\begin{table}
    \begin{tabular}{l r r r r r r}
                    & \textbf{TU's} & \textbf{BMIs} & \textbf{Scan} & \textbf{Build} & \textbf{Total} & \textbf{Space} (MiB) \\
      Infinity      & 1\,485        & 643           & 7.00          & 2:00           & 1:12:42        & 2671.9               \\
      async\_simple & 87            & 64            & 0.45          & 0:05           & 1:12           & 138.6
    \end{tabular}
  \end{table}
\end{frame}


\begin{frame}{We ended up having to}
  \begin{enumerate}
    \item Scan the source files to find \texttt{import} / \texttt{export} statements.
    \item Build the dependency graph.
    \item Traverse it in order to generate the BMIs.
    \item On incremental changes, implement the logic to:
          \begin{itemize}
            \item Rebuild what is necessary to analyze the modified file.
            \item Reanalyze what is affected by the changes.
            \item Rebuild what is necessary to analyze the files affected by the change \emoji{grimacing-face}.
          \end{itemize}
    \item Do all of this while trying to utilize the available cores efficiently.
  \end{enumerate}
  If we had code generation, we could almost build the program.\footnote{\tiny Just kidding, this would only half-work for module-only projects.}
\end{frame}

\begin{frame}{One takeaway?}
  \begin{figure}
    \includegraphics[width=0.5\linewidth]{questions.png}
  \end{figure}
  \centering
  Probably yes.
\end{frame}

\section*{What About Ill-Formed Programs?}
% Wait, What? Why?
% Unsupported compilers / builtins (not much of a problem for the big three that support modules)
% AutoScan!
% We need to write BMIs from Ill-formed code
% Turns out clang is pretty resilient to this, but not perfect, quite a few crashes
%   Do they care? Maybe they do (clang-tidy, clangd)
%   Tooling probably cares in general about this aspect
\begin{frame}{Wait... what?}
  \begin{block}{}
    Analyzers need to be more resilient to parsing errors.
    \begin{itemize}
      \item Unsupported compilers or unsupported builtins.
      \item Unsupported language features.
      \item SonarQube for IDE.
    \end{itemize}
  \end{block}
\end{frame}

\begin{frame}{Wait... what?}
  \begin{block}{\href{https://docs.sonarsource.com/sonarqube-cloud/advanced-setup/automatic-analysis/\#automatic-analysis-for-c-and-c-projects-}{Automatic Analysis}}
    \begin{itemize}
      \item No configuration needed.
      \item Simply link SonarQube to your repository to get started.
      \item Ideal for small to medium-sized projects.
      \item A full talk on its own.
    \end{itemize}
  \end{block}
\end{frame}

\begin{frame}[fragile]{Wait... what?}
  Clang is already quite robust and can generate an AST even for ill-formed programs.
  \vspace{1em}
  \begin{columns}
    \begin{column}{0.4\textwidth}
      \begin{minted}{cpp}
int foo() {
    int i = 0;
    if (IForgotAnInclude()) {
        i += 1;
    }
    return i;
}
    \end{minted}
    \end{column}
    \begin{column}{0.6\textwidth}
      \tiny\texttt{
      TranslationUnitDecl \\
      `-FunctionDecl <line:2:1, line:8:1> line:2:5 foo 'int ()' \\
      `-CompoundStmt <col:11, line:8:1> \\
      |-DeclStmt <line:3:5, col:14> \\
      | `-VarDecl <col:5, col:13> col:9 used i 'int' cinit \\
      |   `-IntegerLiteral <col:13> 'int' 0 \\
      |-IfStmt <line:4:5, line:6:5> \\
      | |-{\color{red}RecoveryExpr <line:4:9, col:26> '<dependent type>' contains-errors} lvalue \\
      | | `-{\color{purple}UnresolvedLookupExpr <col:9> '<overloaded function type>' lvalue (ADL) = 'IForgotAnInclude' empty} \\
      | `-CompoundStmt <col:29, line:6:5> \\
      |   `-CompoundAssignOperator <line:5:9, col:14> 'int' lvalue '+=' ComputeLHSTy='int' ComputeResultTy='int' \\
      |     |-DeclRefExpr <col:9> 'int' lvalue Var 0x2a9dbc90 'i' 'int' \\
      |     `-IntegerLiteral <col:14> 'int' 1 \\
      `-ReturnStmt <line:7:5, col:12> \\
      `-ImplicitCastExpr <col:12> 'int' <LValueToRValue> \\
      `-DeclRefExpr <col:12> 'int' lvalue Var 0x2a9dbc90 'i' 'int'
      }
    \end{column}
  \end{columns}
  \vspace{0.5em}
  \begin{center}
    \small What if this was a module with an export?
  \end{center}
\end{frame}

\begin{frame}[t]{Modules with errors}
  \begin{block}{}
    \small Fortunately clang can emit and load BMIs with errors: \texttt{-fallow-pcm-with-compiler-errors}\footnote{Capability that was \href{https://github.com/llvm/llvm-project/pull/121485}{briefly gone between versions 19 and 20}.}.

    \vspace{0.5em}

    Unfortunately, there were some bugs lurking around, and there may still be:
    \begin{itemize}
      \item \href{https://github.com/llvm/llvm-project/pull/111179}{\#111179 Don't evaluate concept when its definition is invalid}
      \item \href{https://github.com/llvm/llvm-project/pull/121550}{\#121550 Do not serialize function definitions without a body.}
      \item \href{https://github.com/llvm/llvm-project/pull/121768}{\#121768 Fix initialization of NonTypeTemplateParmDecl when there are invalid constraints.}
    \end{itemize}
  \end{block}
  \begin{block}{}
    \small Not all bugs found were due to modules \emoji{sweat-smile}:
    \href{https://github.com/llvm/llvm-project/pull/118288}{\#118288 Fix non-deterministic infinite recursion...} (mp-units)
  \end{block}
\end{frame}

\begin{frame}{What is missing?}
  \begin{block}{}
    \begin{enumerate}
      \item \texttt{import std;} for Automatic Analysis.
      \item Review rules that can be affected by modules. Some known \emph{False Positives}
            \begin{itemize}
              \item \href{https://sonarsource.github.io/rspec/\#/rspec/S1000/cfamily}{S1000}: Header files should not contain unnamed namespaces
              \item \href{https://sonarsource.github.io/rspec/\#/rspec/S1003/cfamily}{S1003}: \texttt{using namespace} directives should not be used in header files
            \end{itemize}
      \item SonarQube for IDE.
      \item Allow users to manage the caching of BMIs.
      \item Handling multiple builds of a module within a single analysis.
      \item Header units.
    \end{enumerate}
    Adoption is still low, so hard to prioritize.
  \end{block}
\end{frame}

% Where is it supported?
\begin{frame}{Support}
  \begin{block}{}
    \begin{table}
      \begin{tabular}{m{10em}  p{3em}}
        \includesvg[width=8em]{logos/SQ_Logo_Cloud_Light Backgrounds}  & \emoji{green-circle} \\[1em]
        \includesvg[width=8em]{logos/SQ_Logo_Server_Light Backgrounds} & \emoji{green-circle}
      \end{tabular}
    \end{table}
  \end{block}
  \begin{block}{}
    \begin{table}
      \newcolumntype{Y}{>{\centering\arraybackslash}X}
      \begin{tabular}{m{10em} p{3em}}
        \includesvg[width=8em]{logos/SQ_Logo_IDE_Light Backgrounds} & \emoji{red-circle} \\[1.5em]
        \multicolumn{2}{c}{
          \begin{tabularx}{14em}{l Y Y r}
            \includesvg[width=1.5em]{logos/Clion} & \includesvg[width=1.5em]{logos/vscode} & \includesvg[width=1.5em]{logos/vs} & \includesvg[width=1.5em]{logos/eclipse} \\
            \multicolumn{4}{l}{} Different IDEs, different APIs                                                                                                           \\
          \end{tabularx}
        }
      \end{tabular}
    \end{table}
  \end{block}
\end{frame}

\begin{frame}{}
  Special mention to Tomasz Kaminski, who led the effort and
  submitted a Defect Report:
  \begin{itemize}
    \item \href{https://cplusplus.github.io/CWG/issues/2921.html}{CWG 2921}
    \item Refined in \href{https://cplusplus.github.io/CWG/issues/2990.html}{CWG 2990}
  \end{itemize}
\end{frame}

\begin{frame}[standout]
  \begin{block}{{\color{white}Key Points}}
    \small
    \begin{itemize}
      \item C++20 module support is becoming more stable across major compilers.
      \item Integrating module support into the analyzer required a significant effort.
      \item The adoption rate of modules is still relatively low.
      \item If you try using modules and analyze them, your feedback would be greatly appreciated.
    \end{itemize}
    \vspace{1em}
    \centering
    \begin{tabular}{m{2em} p{18em}}
      \includesvg[width=2em]{logos/SonarMarkDark.svg} & \href{https://community.sonarsource.com/}{\color{cyan}https://community.sonarsource.com/}
    \end{tabular}

  \end{block}
  \begin{block}{}
    \centering
    Questions?
  \end{block}
\end{frame}

\begin{frame}{More Materials}
  \printbibliography[heading=none]
\end{frame}

\end{document}
